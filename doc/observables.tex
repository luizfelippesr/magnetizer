\documentclass[a4paper,10pt]{article}
\usepackage{amsmath}
\usepackage{amsfonts}
\usepackage{amssymb}
\usepackage{bm}
\usepackage[utf8]{inputenc}

\newcommand{\ee}{\mathrm{e}}  %Euler's constant
\newcommand{\dd}{\mathrm{d}}  %diferential
\renewcommand{\vec}{\bm} % makes bold vectors

\setlength{\parindent}{0pt}
\setlength{\parskip}{10pt}
%opening
\title{Observables}
\author{Luiz F. S. Rodrigues}
%
\begin{document}

\section{Construction of LoS quantities}
\subsection{Coordinates}

In the original output we have the cylindrical radial coordinates: $s_1 .. s_N$.

These are translated into coordinates in a line of sight (LoS) using the following
procedure.
Let $\theta$ be the angle between the LoS and a normal to the galactic disc.
We assume that the LoS is parallel to the $xz$ plane. The LoS can therefore be
characterised by a choice of $\theta$, and and choosing the point where the LoS
intercepts the $yz$ plane, $(y, z)|_\text{intercept} \equiv(b_y,b_z)$.


Now we construct $2N$ points using the following:

If $i\leq N$, $j = N-i+1$ and
\[
    x_{i} = -\sqrt{s_j^2 - b_y^2}
\]
But if $i>N$, $j = i-N$ and
\[
    x_{i} = \sqrt{s_j^2 - b_y^2}
\]



From our knowledge of $x_i$, we can construct the vertical coordinate:
\[
    z_i = \frac{x_i}{\tan\theta} + b_z
\]

Thus, the $2N$ LoS points are given by $(x_i,b_y, z_i)$.

Any values of $i$ where ${s_j^2}<b_y^2$ are marked as invalid, and thus skipped
and later removed from the calculation before the integrations.
The removal is done using \textsc{Fortran} 95's \texttt{pack}
function, which, given an array and a mask, outputs an array of smaller length.

\subsection{Density}

Using the same conventions for $i$ and $j$ as previously, the density in the output,
$n^0$, can be translated into the LoS-density, $n$, using

\[
    n_i = n^0_j \times\ee^{-|z_i|/h^0_j}
\]
where $h^0$ is the scale-height in the output.

\clearpage
\subsection{Magnetic field}

We have $N$ values of $B^0_r$, $B^0_\phi$, $|B^0_z|$
on the mid-plane.

To compute the $x$-, $y$- and $z$-components over the LoS, we do
\[
\widetilde B_{x,i} = B^0_{r,j} \frac{x_i}{s_j} - B^0_{\phi,j} \frac{b_y}{s_j}
\]
\[
\widetilde B_{y,i} = B^0_{r,j} \frac{b_y}{s_j} + B^0_{\phi,j} \frac{x_i}{s_j}
\]
\[
\widetilde B_{z,i} = |B^0_{z,j}| \times \text{sign}(z_i)
\]
NB $x/s = \sin\phi$ and $b_y/s = \cos\phi$

Then, the $z$-dependence is added. There are two options, so far: scaling the same way
as the gas,
\[
    B_{x/y/z,i} = \widetilde B_{x/y/z,i} \exp(-|z_i|/h^0_j)
\]
or assuming constant values within a scale-height
\[
    B_{x/y/z,i} = \begin{cases}
                    \widetilde B_{x/y/z,i}\,, & |z_i| \leq h^0_j\\
                                0\,,          & |z_i| >    h^0_j
                  \end{cases}
\]

From these components, the projections are computed, using the initial $\theta$ value,
since the LoS is parallel to $\vec {\hat n} = (\cos\theta, 0, \sin\theta)$, the projection
$B_\parallel = \vec B \cdot \vec{\hat n}\,$ is given by
\[
    B_\parallel = B_x \cos\theta + B_z \sin\theta,
\]

From this we can get the angle between $\vec B$ and the LoS
\[
    \beta = \arccos(B_\parallel/|\vec B|)
\]
and compute the perpendicular component
\[
    B_\perp = |\vec B|\sin\beta
\]

\section{Observables}

All the integrations are, for the moment, performed using a simple trapezoidal rule
(this can be substituted by something else, if needed).
Thus, for each 2 grid points in the LoS path,
\[
 I_i = \frac{F_{i+1}+F_{i}}{2} \Delta l\,,
\]
with
\[
    \Delta l = \sqrt{ (x_{i+1}-x_i)^2 + (z_{i+1}-z_i)^2}\,,
\]
therefore, the total integral is
\[
 I = \sum_{i}^{N-1} I_i
\]




\subsection{RM}
\[
    \int 0.812 B_\parallel n_e \,\dd l
\]

Where we take $n_e=n$ (i.e. .
\subsection{Emissivity}

\[
 \varepsilon = n_{cr} B_\perp^{(\alpha+1)/2}\lambda^{(\alpha-1)/2}
\]

We take (at least provisionally) $n_{cr} = n$ since $n_{cr}\propto b^2 \propto n$.

\subsection{Stokes I}
\[
    \int \varepsilon  \,\dd l
\]


\end{document}
