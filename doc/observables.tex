\documentclass[a4paper,10pt]{article}
\usepackage{amsmath}
\usepackage{amsfonts}
\usepackage{amssymb}
\usepackage{bm}
\usepackage[utf8]{inputenc}

\newcommand{\ee}{\mathrm{e}}  %Euler's constant
\newcommand{\dd}{\mathrm{d}}  %diferential
\renewcommand{\vec}{\bm} % makes bold vectors
\newcommand{\degree}{^\circ}
\newcommand{\kpc}{\,\text{kpc}}  %kpc|
\newcommand{\muG}{\,\mu{\rm G}}  %microgauss
\newcommand{\cm}{\,{\rm cm}}

\setlength{\parindent}{0pt}
\setlength{\parskip}{10pt}
%opening
\title{Observables}
\author{Luiz F. S. Rodrigues}
%
\begin{document}

\section{Construction of LoS quantities}
\subsection{Coordinates}

\subsubsection{Sparse LoS-grid}

In the original output we have the cylindrical radial coordinates: $s_1 .. s_N$.

These are translated into coordinates in a line of sight (LoS) using the following
procedure.
Let $\theta$ be the angle between the LoS and a normal to the galactic disc.
We assume that the LoS is parallel to the $xz$ plane. The LoS can therefore be
characterised by a choice of $\theta$, and and choosing the point where the LoS
intercepts the $yz$ plane, $(y, z)|_\text{intercept} \equiv(b_y,b_z)$.


Now we construct $2N$ points using the following:

If $i\leq N$, $j = N-i+1$ and
\begin{equation}
    \widetilde{x}_{i} = -\sqrt{s_j^2 - b_y^2}
\end{equation}
But if $i>N$, $j = i-N$ and
\begin{equation}
    \widetilde x_{i} = \sqrt{s_j^2 - b_y^2}
\end{equation}



From our knowledge of $\widetilde x_i$, we can construct the vertical coordinate:
\begin{equation}
    \widetilde z_i = \frac{\widetilde x_i}{\tan\theta} + b_z
\end{equation}

Thus, the $2N$ LoS points are given by $(\widetilde x_i,b_y, \widetilde z_i)$.

Any values of $i$ where ${s_j^2}<b_y^2$ are marked as invalid, and thus skipped
and later removed from the calculation just after the sparse LoS-grid.
The removal is done using \textsc{Fortran} 95's \texttt{pack}
function, which, given an array and a mask, outputs an array of smaller length.


\subsubsection{Dense/interpolated LoS-grid}

Working with the sparse LoS grid constructed in the previous sub-section works fine
for (almost) edge-on galaxies. However, for (almost) face-on, it is problematic,
as it will be illustrated below.

Imagine we have a spacing $\Delta R=1\kpc$ and a LoS intercepting the mid-plane at
an angle $\theta=1\degree$, this will correspond a distance of $\Delta z = 57\kpc$
between the grid constructed over LoS. Because most quantities scale
$\propto\exp(-|z|/h)$, this will lead to (artificially) $\approx\!0$ valued
quantities in all our calculations.

To tackle this, we construct a new grid, with fixed number of points $N_\text{dense}$
along the LoS, linearly spaced\footnote{A different spacing may be useful --- e.g.
to account for the exponential decay in $z$ --- and will probably be used in the future.
The reason for keeping it linear for now is the fact that $h$ depends itself strongly on
$\widetilde x$, making it harder to work out the optimal spacing.}
in $z$ with $z_\text{min}<z<z_\text{max}$. We use $4\times$
the scale-height as a limit (unless this is outside the available points),
\begin{equation}
 z_\text{min} = \max\left[\widetilde z_0,\; -4\times \widetilde h_0\right]
 \quad\text{and}\quad
 z_\text{max} = \min\left[\widetilde z_{2N},\; 4\times \widetilde h_{2N}\right]
\end{equation}
with $\widetilde h_i = h(\widetilde x_i) = h^\text{o}_j$ (the scale-height at position $j$ in
the original output).

All the quantities are computed in this new grid. The procedure is the following:
first, the mid-plane values of the quantity, $q$, are found on the LoS
\begin{equation}
  \widetilde q_i = q^{o}_j
\end{equation}
this is linearly interpolated (in $x$) to get the value on a point of the new grid
\begin{equation}
 q'_k =  \widetilde q_i + \left(\widetilde q_{i+1}-\widetilde q_i\right) \frac{x_k - \widetilde x_i}{\widetilde  x_{i+1}-\widetilde x_i}
%  y(j) + (y(j+1)-y(j)) * (xi(i) - x(j))/(x(j+1) - x(j))
\label{eq:interp}
\end{equation}
with $\widetilde x_i < x_k < \widetilde x_{i+1}$.

Only after this, the $z$-dependence is included,
\begin{equation}
 q_k = F(q'_k, h_k, z_k)
\end{equation}
where the function $F$ will depend on the quantity and $h_k=h'_k$ is the interpolated
scale-height.





\subsection{Density}

Using the same conventions for $i$ and $j$ as previously, the density in the output,
$n^\text{o}$, can be translated into the LoS-density, $n$, using

\begin{equation}
    \widetilde n_i = n^\text{o}_j\,
\end{equation}
$n'_k$ is computed from Eq.~\eqref{eq:interp} and
\begin{equation}
 n_k = n'_k \times \ee^{-|z_k|/h'_k}
\end{equation}


\subsection{Magnetic field}

We have $N$ values of $B^\text{o}_r$, $B^\text{o}_\phi$, $|B^\text{o}_z|$
on the mid-plane.

To compute the $x$-, $y$- and $z$-components over the LoS, we do
\begin{equation}
\widetilde B_{x,i} = B^\text{o}_{r,j} \frac{x_i}{s_j} - B^\text{o}_{\phi,j} \frac{b_y}{s_j}
\end{equation}
\begin{equation}
\widetilde B_{y,i} = B^\text{o}_{r,j} \frac{b_y}{s_j} + B^\text{o}_{\phi,j} \frac{x_i}{s_j}
\end{equation}
\begin{equation}
\widetilde B_{z,i} = |B^\text{o}_{z,j}| \times \text{sign}(z_i)
\end{equation}
% \begin{equation}
% \widetilde B_{x,i} = B^\text{o}_{r,j} \cos\phi - B^\text{o}_{\phi,j} \sin\phi
% \end{equation}
% \begin{equation}
% \widetilde B_{y,i} = B^\text{o}_{r,j} \sin\phi + B^\text{o}_{\phi,j} \cos\phi
% \end{equation}
NB $x/s = \cos\phi$ and $b_y/s = \sin\phi$



From these components, the projections are computed, using the initial $\theta$ value,
since the LoS is parallel to $\vec {\hat l} = (\sin\theta, 0, \cos\theta)$, the projection
$B_\parallel = \vec B \cdot \vec{\hat l}\,$ is given by
\begin{equation}
    \widetilde  B_\parallel = B_x \sin\theta + B_z \cos\theta
    \quad \text{and}\quad
    \widetilde {\vec B}_\parallel = \widetilde B_\parallel \vec {\hat l}
\end{equation}
and the perpendicular component is given by
\begin{equation}
    \vec B_\perp = \vec B -\vec B_\parallel
\end{equation}
or
% \begin{equation}
%  \vec B_\perp = \left( B_x - B_\parallel\sin\theta\,,
%                  \; B_y\,,
%                 \;  B_z -B_\parallel\cos\theta\right)
% \end{equation}
\begin{equation}
 \widetilde B_\perp = \sqrt{ (\widetilde B_x - B_\parallel\sin\theta)^2 +
                 \widetilde B_y^2 +
                 (\widetilde B_z -\widetilde B_\parallel\cos\theta)^2}
\end{equation}
% \begin{align}
% \vec B_\perp =& \left[ B_x \cos^2\theta - B_z \sin\theta\cos\theta\right] \hat{\vec i}
%                 + B_y \,\hat{\vec j}\\
%                & + \left[ B_z \sin^2\theta - B_x \sin\theta\cos\theta\right] \hat{\vec k}
% \end{align}

At this point, we do the interpolation into the ``dense'' LoS grid using Eq.~\ref{eq:interp},
i.e. $\widetilde B_\parallel \rightarrow B'_\parallel$ and
 $\widetilde B_\perp \rightarrow B'_\perp$ and then add the $z$-dependence is added.
 There are two options, so far: scaling the same way as the gas,
\begin{equation}
    B_{\parallel/\perp, k} = B'_{\parallel/\perp, k} \exp(-|z_k|/h_k)
\end{equation}
% \begin{equation}
%     B_{\parallel, k} = B'_{\parallel, k} \exp(-|z_k|/h_k)
% \end{equation}
% \begin{equation}
%     B_{\perp, k} = B'_{\perp, k} \exp(-|z_k|/h_k)
% \end{equation}
or assuming constant values within a scale-height
\begin{equation}
    B_{{\parallel/\perp},k} = \begin{cases}
                    B'_{{\parallel/\perp},k}\,, & |z_k| \leq h_k\\
                      0\,,          & |z_k| >    h_k
                  \end{cases}
\end{equation}

\section{Observables}

All the integrations are, for the moment, performed using a simple trapezoidal rule
(this can be substituted by something else, if needed).
Thus, for each 2 grid points in the LoS path,
\begin{equation}
 I_i = \frac{F_{i+1}+F_{i}}{2} \Delta l\,,
\end{equation}
with
\begin{equation}
    \Delta l = \sqrt{ (x_{i+1}-x_i)^2 + (z_{i+1}-z_i)^2}\,,
\end{equation}
therefore, the total integral is
\begin{equation}
 I = \sum_{i}^{N-1} I_i
\end{equation}

The number of points used is $N_\text{dense}$ (by default, this was arbitrarily chosen
to be 500).




\subsection{RM}
\begin{equation}
    \left(812 \,\text{rad\,m}^{-2}\right) \times  \int{ \frac{B_\parallel}{1 \muG} \frac{n_e}{1 \cm^{-3}} \,\frac{\dd l}{1\kpc}}
\end{equation}

Where we take $n_e=n$ (i.e. .
\subsection{Emissivity}

\begin{equation}
 \varepsilon = n_{cr} B_\perp^{(\alpha+1)/2}\lambda^{(\alpha-1)/2}
\end{equation}

We take (at least provisionally) $n_{cr} = n$ since $n_{cr}\propto b^2 \propto n$.

\subsection{Stokes I}
\begin{equation}
    \int \varepsilon  \,\dd l
\end{equation}


\end{document}
